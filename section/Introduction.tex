\chapter{Introduction}

\section{Problemstellung}
In der Problemstellung erfolgt eine Hinführung zum Thema aus dem globalen Zusammenhang heraus betrachtet: Warum es wichtig ist, sich mit dem konkreten Thema zu beschäftigen bzw. welche Bedeutung hat dieses Thema zum Beispiel für die Wirtschaft, Gesellschaft und Umwelt.

\section{Zielsetzung und wissenschaftliche Fragestellung}
Hier werden die konkrete(n)wissenschaftliche(n) Fragestellung(en) klar angeführt. Neben dem inhaltlichen Ziel der Arbeit wird fallweise auch angegeben, wie vorgegangen wird, um die Fragestellung zu bearbeiten (z.B. mit einem Online-Fragebogen). Auch die Zielgruppe der Arbeit und der für die Zielgruppe angestrebte Nutzen kann an dieser Stelle angeführt werden.
Anmerkung: 
Die Erfahrung zeigt, dass es hilfreich ist, das erste Einleitungskapitel an die Betreuerin / an den Betreuer zu schicken und Feedback dazu zu erhalten.
Bitte auch darauf achten, dass alle Absatz-Abstände gleich groß sind.
Jedes Kapitel sollte mit einem Satz beginnen und einem Satz enden und ca. mind. \sfrac{1}{2} A4-Seite umfassen.
\ac{aws} vfvdv \ac{aws} vfvbb.

%Zitat \autocite[S. 20]{diekmann2010}


